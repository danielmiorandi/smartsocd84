\subsection{Context Manager}
\subsubsection{Functionality and Features}

The Context Manager (CM) is the software/hardware embodiment of the three-layer approach for context recognition described in~\cite{D4.3}. It includes three components:
\begin{itemize}
\item Sensing devices: able to capture relevant parameters for identifying the peer context.
\item Modules for aggregating and merging attribute values from sensor data.
\item High level models stored in each peer and connected via dedicated APIs.
\end{itemize}
The CM works as an extension of the PM and is used by the TEM to monitor parameters relevant to the task execution status and progress. The low level data collected from the users' mobile device and stored into the sensor stream database are analysed by the CM and pushed to the attributes of the corresponding agents in the PM. When the Task Execution Manager requests to monitor a specific event, the Context Manager analyses the attributes of the people involved in the requested event in the PM and generates higher level status information. In the current implementation we are able to track if a Travel is started, ended and in execution.

Every event can be monitored using a monitoring process that has to be registered to the CM and unregistered when no new updates are needed any more. When the status for a travel updates because of new sensor data arriving to the servers, the CM pushes the corresponding information to the Task Execution Manager as specified in their APIs, using the PUT method /updateMonitor/:rideRequestID. At the same time, the Context Manager provides a pull functionality according to which the component who registered a monitor can ask for its status at any time by calling a specific endpoint.

\subsubsection{Implementation}

The CM is written in multiple languages depending on the component. 
\begin{itemize}
\item{The mobile iLog application that collects sensor data from the users' mobile device(s) is written in Java using the Android SDK and the Spring Library for communicating with the Server.}
\item{The big data storage system that collects the sensor streams from the iLog application is written in Java. At the core of the system we use Cassandra distributed storage system that provides linear scalability by adding machines to the cluster.}
\item{The big data analytics system that aggregates and merges the streams and generates the abstractions to be pushed to the PM is designed using the Apache Spark distributed computational framework. As well as for the Cassandra system, linear scalability can be reached by adding machines to the cluster.}
\item{Finally, the API exposed by the CM to monitor a task are written in JavaScript on top of Node.js .}
\end{itemize}

\subsubsection{Interfaces, Endpoints and Resources Exposed}

\begin{itemize}
\item {\bf POST /monitorTask} Starts the process that monitors a specific task that as to be passed in the body of the request.
\item {\bf GET /monitorTask} Get all the tasks which are being monitored by TEM with their status.
\item {\bf GET /monitorTask/:rideRequestID} Get a specific task which is being monitored by TEM with its status.
\item {\bf DELETE /monitorTask/:rideRequestID} Terminate the monitoring of a task.
\end{itemize}

\subsubsection{Repository}
Not applicable, the CM is owned by the University of Trento and not disclosed. 
