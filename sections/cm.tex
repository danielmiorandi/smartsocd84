\subsection{Context Manager}
\subsubsection{Functionality and Features}
\todo{@Mattia/Ronald: please check}
The Context Manager (CM) is the software/hardware embodiment of the three-layer approach for context recognition described in~\cite{D4.3}. It includes three components:
\begin{itemize}
\item Sensing devices: able to capture relevant parameters for identifying the peer context.
\item Modules for aggregating and merging attribute values from sensor data 
\item High level models stored in each peer and connected via dedicated APIs
\end{itemize}
The CM works as an extension of the PM and is used by the TEM to monitor parameters relevant to the task execution status and progress.
\subsubsection{Implementation}
Not applicable, the CM is owned by University of Trento and implementation details are not made public.

\subsubsection{Interfaces, Endpoints and Resources Exposed}
\todo{@Mattia/Ronald: please check (add interfaces used by the PM?)}
\begin{itemize}
\item {\bf POST /monitorTask} Starts the process that monitors a specific task that as to be passed in the body of the request.
\item {\bf GET /monitorTask} Get all the tasks which are being monitored by TEM with their status.
\item {\bf GET /monitorTask/:taskID} Get a specific task which is being monitored by TEM with its status.
\item {\bf DELETE /monitorTask/:taskID} Terminate the monitoring of a task.
\end{itemize}
\subsubsection{Repository}
Not applicable, the CM is owned by the University of Trento and not disclosed. 