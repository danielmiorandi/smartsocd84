When back in 2012 the Consortium started working on the ideas and concepts that later on became the SmartSociety project the idea of a ``platform for social computations'' was meant to play a threefold role:
\begin{itemize}
\item To balance the flavour of R\&D\&I activities to be carried out in the project, making sure Consortium partners would not focus just on the abstract science of building and managing HDA-CASs, but would also consider constraints and barriers coming from the need to actually implement a working prototype thereof.
\item To ensure consistency of the SmartSociety vision and outputs, acting as a single integration point for the whole Consortium activities~\footnote{This role is actually shared with WP1, which however focuses more on the ethical aspects and societal impacts of HDA-CASs.}.
\item To respond to the need of having a computing infrastructure able to support HDA-CASs, to be used (first) by researchers for experimenting with CAS concepts and practises and (in the long term) to act as technology enabler for a new generation of ICT systems able to truly account for the human and social dimensions of computation in the real world.
\end{itemize}  

This vision evolved over time as activities progressed and the Consortium gained more insight into the building blocks required to make a HDA-CASs working. The net output, which is presented in the deliverable at hand, should be understood as a prototypical implementation of a future computing infrastructure able to leverage hybrid, diversity-rich collectives to carry out complex computational tasks (possibly spanning the virtual and physical world).

The SmartSociety platform is --- to a large extent --- released under a permissive open source license\footnote{Only the PM and the CM are not publicly released, yet they APIs are, so that a developer skilled in the art could work out her bespoke implementation of the functionality and have it promptly integrated with the rest of the platform.}. While this is meant to ensure liveness of the software artifacts being developed, it should be coupled with an appropriate strategy for ensuring a coherent exploitation of the project's results. At the moment the overall platform can be considered somewhere between TRL-3 and TRL-4,\footnote{It has to be noted that some components have actually higher TRL, for example \mdl is considered TRL-6.} hence major efforts are needed to turn it into a product ready to hit the market. Yet, all the basic building blocks required to successfully execute hybrid computation at scale are there, so this is understood to represent a good starting point for anybody (scientist, startups and Web entrepreneurs, industries) active in the field.

At the time of writing an open source server model, similar to the one adopted by Prediction.io\footnote{\url{https://prediction.io/}}, seems to be the most viable solutions for ensuring a sustainable exploitation of the platform codebase: more details will be provided in the final project exploitation plan. One of the reason underpinning this choice relates to the level of generality of the solutions developed. While indeed the project identified a number of recurrent design patterns that can be applied to a large variety of social computations, there are still aspects that depend on the specific context the social computation is embedded in. This requires, in turn, to tailor some of the methods implemented by platform components (e.g., incentives, composition, negotiation etc.) to the specific use case. This is ---we do believe--- one of the key lessons learned by the Consortium: while at the beginning we were hoping to achieve full generality in terms of enabling technologies for HDA-CAS, we discovered that this is not realistic in practice. As no one-size-fits-all solution can be developed, the choice of the open source server provides designers and developers with a set of useful building block, which they can tailor to the specific realm they target with their application, thereby delivering clear benefits (in terms of reduced entry barriers in terms of required knowledge, time and development cost) without restricting too much the scope of applicability. 
% In terms of major changes in perspective, a relevant one is the shift from a platform model to an open source server one. 


% The vision of the SmartSociety platform is to become an `IFTTT~\footnote{{\tt https://ifttt.com/}: IFTTT (IF This Then That) is a Web-based service allowing non-technical users to create chains of conditional statements (called `IF recipes') and actions (called `DO recipes') involving popular Web services.} for social computation', i.e., a platform that allows the easy integration of hybrid (human and machine) computational elements within the scope of a well-defined application workflow. While various steps have been taken in this direction, the overarching goal will require intense activities for the remainder of the project. 

% Version 2.0 of the SmartSociety platform integrates seven key components: peer manager, orchestration manager, communication middleware, provenance service, reputation service, monitoring and analysis service and application runtime. This configuration is sufficient for developing a number of different applications using various types of social computation. Some limitations in terms of the single components are expected to be overcome with refinements from the relevant technical WP.

% In terms of the upcoming steps, four major components to be integrated are the context manager (which will interact strictly with the peer manager), the reputation maanager, the task execution manager and the incentive server. Major changes in the platform configuration will come from the delivery of the programming framework by WP7 (foreseen at M36), which will have a major impact on how platform functionality can be accessed by application developers. Interactions with the virtual gamified environment in WP9 may also lead to a refactoring of some of the features exposed by the platform. Another major release of the platform is foreseen at M36. 

Something that is worth highlighting in this section is one issue that arose while discussing with WP1 team (see also~\cite{D1.2} for a more extended discussion) on the ethical implications of some of the design choices that were taken by the project's technical team. When dealing with HDA-CASs indeed it turns out that design decisions in terms of enabling technology are at the centre of three often conflicting forces: the technical aspects (with the technical team typically interested in shipping high-performance and spotless code), the business aspect (where the choice in terms of exploitation model and business plan can have a major impact on how the codebase develops) but, and this is pretty peculiar to HDA-CASs, also ethical ones (where adherence to a responsible research and innovation approach does impact, often heavily, choices in terms of technology and exploitation model). To give a concrete example: WP1 put forward the need for a HDA-CASs to have a `constitution' highlighting the values underpinning the CASs and the rights of all types of participants. Technically, this would mean that (referring to the PF primitives) when a CBT is instantiated it should have such constitution declared (which should be passed on to the CBTBuilder); in other words, some changes to the code are required. But if we decide to release the platform as an open source server, anybody could escape such steps by just tweaking the code, so this seems to also impact the business model rather significantly. At the same time if we decide to go for a PaaS model, which would allow us --- as platform owners --- to tightly control the constitution aspect, this would imply some major modifications to the codebase, which in turn would impact the financial perspectives in terms of launching a new venture. The tight entanglement of these three aspects (ethics, technology and business) appears to be probably the most challenging aspect of HDA-CASs, and certainly one of the major take-aways of the project. 

% \todo{Taken from D1.2}
% There is here an interesting element related to the deployment (and business) model of the platform. If the platform is made available 'as a service', indeed, the platform controller can exercise some form of control, for example by means of a code of conduct (formalised through, e.g., terms and conditions for the usage of the platform services). This could be use to perform some control on what the platform is used for, and eventually ban some applications which are deemed not in line with the ethical principles of the SmartSociety project. On the other hand by providing the platform as an open-source server that anybody can setup and deploy there are no means to enforce any form of control, so that the platform could, at least theoretically, be used for managing and operating 'evil' (read it: discriminatory, non democratic etc.) HDA- CASs. There is therefore a tension here, in that both models bring about advantages and disadvantages, which should be properly balanced. 
%In line with the DoW, the actual internal release of the first version of the platform (including user modules) represents MS18 and is due at M27. By M27 the integration roadmap foresees the integration of the provenance store and of the reputation manager. It will further include a fi5rst version of the execution manager and of the monitoring and analysis service. Security and privacy aspects will be integrated during the third year of the project, in particular within the scope of WP4 (where all personal information is stored). The programming framework development will be strictly aligned with the platform functionality enhancement and will support the easy registration and deployment of applications. The context manager will be integrated with the peer manager in the course of 2015. The incentives manager will be integrated in a later stage of the project, the actual integration date depending on the progress of research activities within WP5. Further work will be carried out in a dialogue with WP1 in terms of devising mechanisms supporting various governance models for the platform. Last, but not least, the development of the platform will also reflect the work on exploitation plans and potential business models carried out within the scope of WP10.

%The second platform prototype (including validation results from lab experiments) will be delivered as D8.3 at M30. 

% In this deliverable we have introduced:
% \begin{itemize}
% \item An analysis of requirements for the SmartSociety platform, elicited by working in  cooperation with the other work packages and covering both functional and non-functional properties. 
% \item A system-level architecture, encompassing:
% \begin{itemize}
% \item the definition of the logical and functional role of each platform component;
% \item a definition of the interactions among components;
% \item the preliminary identification of modules within each component, in line with the progress of single technical WPs.
% \end{itemize}
% \end{itemize}

% This document represents the starting point for the integration activities taking place in the second year of the project (T8.2 and T8.3). The overall architecture will be iteratively revised, improved and extended during the second year of the project, leading to the delivery, at M24, of a first prototypical implementation of the platform. D8.1 will
% serve as reference document and handbook for the development and prototyping activities to be carried out within WP2-WP7, defining clearly the role of the respective components in the overall platform architecture and the interaction patterns with other components. The architecture will be maintained as a living reference document; the release of the first prototype at M24 will go hand in hand with the provisioning of a revised architectural specification, accounting for the evolution of perspectives during year-2.

% Last, it is worth remarking that WP8 played an important {\it integration}
% role in the first year of activities of the SmartSociety project. In order to reach
% a consistent view on the architecture of the SmartSociety platform, indeed,
% WP8 established and fostered an active and open discussion among the
% technical WPs (WP2-WP7). The agreement reached among WPs in terms of logical role, functionalities of their components, and interaction patterns represents an important contribution of WP8 in year-1. 


