\subsection{Monitoring}
\subsubsection{Functionality and Features}
The goal of the monitoring component is to enable system administrators to monitor the liveliness of the SmartSociety platform components, possibly distributed across multiple servers. Target users of the functionality exposed are therefore system administrators and developers of SmartSociety-based CAS enabled applications and services. The main expected usage is for troubleshooting in case of platform malfunctioning, alerts and warnings. In the long term, it can enable the deployment of self-healing mechanisms.


\subsubsection{Implementation}
The implementation of the monitoring component is based upon three basic components:
\begin{itemize}
\item Modules gathering and publishing the relevant information;
\item Modules consuming the monitoring related information;
\item Information dissemination infrastructure.
\end{itemize}
% In particular, local agents collect specific information regarding the liveliness and health status of each component and communicate with the infrastructure via lightweight clients. The monitoring infrastructure provides scalable support to the collection of local agents feeds. A monitoring dashboard represents the main consumer of monitoring data generated by the agents and dispatched through the monitoring infrastructure.
The implementation shipped with the SmartSociety toolkit~\cite{D8.3} is based on the logstash open source framework\footnote{{\tt http://logstash.net/}}, which presents very good support for collection of logs in various schemas/formats, and has a large number of plugins available for the most widely used commercial frameworks. The logstash forwarders accept data using {\tt gerf} protocol on UDP. The logs are then forwarded through secured channel to the main logstash instance.
 The usage of logstash is coupled with elasticsearch for indexing and persistence. Aggregated and curated logs are stored in the elastic database and can be queried via standard interfaces, enabling technical supporting partners to develop their own ad hoc monitoring dashboard or to integrate with legacy ones. The default option for SmartSociety is to use Kibana\footnote{{\tt https://www.elastic.co/products/kibana}}, a flexible dashboard which supports seamless integration with elasticsearch and presents basic, yet sufficient analytics functionality. The metrics and specific charts can be configured dynamically by the administrator of the platform.
\subsubsection{Interfaces, Endpoints and Resources Exposed}
The Monitoring component functionality is accessible through the Kibana GUI. Elasticsearch Search APIs can be used for integration with legacy visualization dashboards. 
\subsubsection{Repository}
Not applicable.
\todo{Tommaso: create project in gitlab and add docker-compose to the repo}