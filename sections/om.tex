\subsection{Orchestration Manager}
\subsubsection{Functionality and Features}
The OM provides collective composition and negotiation functionality. Two patterns are supported. The first one is sequential: the composition phase (during which a collective is created in order to complete a given task) has to be completed in order for the negotiation phase (in which peers have to agree to carry out the task) to start. In the second pattern (that we call continuous orchestration, see~\cite{D6.2} for a detailed description) the two phases are carried out jointly. The OM takes as input task requests and outputs an agreed execution plan. 
\subsubsection{Implementation}
The OM implementation shipped with the toolbox is written in {\tt JavaScript} and based upon the {\tt node.js} framework. It uses {\tt express} for REST APIs, {\tt jade} as node template and includes a {\tt MongoDB} instance for persistency. Two versions are provided, which instantiate the two aforementioned patterns\footnote{They have been used for validation with AskSmartSociety! and SmartShare applications.}. 
They can be used out of the box or can act as template in case specific composition or negotiation algorithms are required. %easily reused to develop application-specific OMs.
\subsubsection{Interfaces, Endpoints and Resources Exposed}
% \todo{@Zhenyu/Michael: please check}
We start with task requests:
\begin{itemize}
\item {\bf POST /applications/:app/taskRequests} Creates a new task request. The JSON object describing the task request is expected in the body of the request. %On success a platform call to the composition manager will be made.
\item {\bf GET /applications/:app/taskRequests/?user=:user} Get all task requests posted by a given user.
\item {\bf GET /applications/:app/taskRequests/:taskRequestID} Get details on a given task request.
\item {\bf HEAD /applications/:app/taskRequests/:taskRequestID} Get the head of a given task request.
\item {\bf GET /applications/:app/taskRequests/:taskRequestID/v/:version} Get a specific version of a given task request.
\item {\bf DELETE /applications/:app/taskRequests/:taskRequestID} Delete a task request.
\end{itemize}
Tasks are generated through composition. The most basic operations related to them are listed in the following table:
\begin{itemize}
\item {\bf GET /applications/:app/tasks/:taskID} Get a specific task (latest version).
\item {\bf HEAD /applications/:app/tasks/:taskID} Get the head of a specific task.
\item {\bf GET /applications/:app/tasks/:taskID/v/:version} Get a specific version of a given task.
\item {\bf PUT applications/:app/tasks/:taskID} Negotiate on a task. The main call for negotiation which will trigger an additional platform call to the negotiation manager. Expects the new version of the document of the task taskID. A platform job for negotiation is prepared and is posted to the negotiation manager. The URI of the new version is returned.
\end{itemize}
Task records are generated by the orchestrator once execution can start on a specific task. The most basic operations are listed in the following table:
\begin{itemize}
\item {\bf GET /applications/:app/taskRecords/:taskRecordID} Get a specific task record.
\item {\bf HEAD /applications/:app/taskRecords/:taskRecordID} Get the head of specific task record. This is another convenience function which allows an easy way for the clients to figure out if the resource has changed, by comparing the response with the latest known version.
\item {\bf GET /applications/:app/taskRecords/:taskRecordID/v/:version} Get a specific version of a given task record.
\item {\bf PUT /applications/:app/taskRecords/:taskRecordID} Provide execution feedback. The main call for execution which will trigger an additional platform call to the execution manager. Expects the new version of the task record document taskRecordID. A platform job for execution is prepared and is posted to the execution manager.
\end{itemize}
The composition manager provides the following functionality:
\begin{itemize}
\item {\bf POST /applications/:app/compositions} Perform composition of collective. Expects the platform job with the description, for which the main ingredient is the new task request that has arrived on the platform. The orchestrator for the application app can make such a call.
\item {\bf GET /applications/:app/compositions/:compositionID} Get composition results, in the form of a collective suitable for carrying out the task. % Normally such a call is expected to happen only once from the application orchestrator once the latter has received the 201 error code that the composition that was requested has been performed.
\end{itemize}
The negotiation manager provides the following functionality:
\begin{itemize}
\item {\bf POST /applications/:app/negotiations} Perform negotiation. Expects the platform job with the description, for which the main ingredient is the task on which negotiation is being performed.
Access Control. The call always succeeds generates a resource describing the outcome of negotiation. Upon completion, it returns an error code 201 and the link to the document with the results of the negotiation. Part of the description of the document with the results of the negotiation is the error code and message that is returned through the call {\bf PUT /applications/:app/tasks/:taskID} to the client.
\item {\bf GET /applications/:app/negotiations/:negotiationID} Get negotiation results. Normally such a call is expected to happen only once from the application orchestrator once the latter has received the 201 error code that the negotiation that was requested has been performed.
\end{itemize}
\subsubsection{Repository}
The OM code is available at: \url{https://gitlab.com/smartsociety/orchestration}