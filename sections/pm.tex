\subsection{Peer Manager}
\subsubsection{Functionality and Features}
The Peer Manager (PM) provides a peer-centered data store that maintains and manages information about human- or machine-based peers within a privacy-preserving framework.
The PM was designed with the objective of keeping the information owned by peers private. In order to provide a flexible management of information, the PM builds upon the notion of an entity-centric semantic enhanced model that defines an extensible set of entity schemas providing the templates for an attribute-based representation of peers’ characteristics~\cite{D4.2}. Additionally, the PM defines a storage and privacy protection model by adding privacy regulations and considerations. The PM is designed to comply with the recent General Data Protection Regulation (GDPR) (Regulation (EU) 2016/679).
\subsubsection{Implementation}
Not applicable, the PM is owned by University of Trento and implementation details are not made public.
\subsubsection{Interfaces, Endpoints and Resources Exposed}
% \todo{add from https://docs.google.com/spreadsheets/d/1UlRKshoALlSzmJYZVLkuzAGTUukdFKaONlKqPmd3RWo/edit#gid=0}
% \todo{@Ronald: please check the APIs \& also whether we can share them (deliverable is PU)}
We divide APIs in chapters. Let us start with the security ones\footnote{We remind hereby that the PM is responsible for authenticating all agents on the platform.}:
\begin{itemize}
\item {\bf POST /security/login} Used for authentication. The call includes as parameters an identifier and a password. In the response a token to be used for further interactions with the platform is included.
\item {\bf POST /security/logout} Log the peer out of the PM.
\item {\bf GET /person} Read the person of a given logged user. 
\item {\bf POST /person\_sl} Register a new human peer. Specifies username and password in the body of the request. Redirects to the newly created peer ID.  %\todo{Ronald: should be a POST? -> Luc is right}
\end{itemize}
The second set of APIs cover management operations:
\begin{itemize}
\item {\bf GET /search}. General search machinery. Check~\cite{D4.3} for more explanations on the query language to be used. 	
\item {\bf POST /collectives} Create a collective: create a collective structure from a set of usernames. Redirects to the created collective.
\item {\bf GET /collectives/:id}	Return the usernames of all peers in a collective.				
\item {\bf DELETE /collectives/:id} Delete the collective with a given :id.	
\item {\bf GET /types} Return the names and id of all the types defined in the peer manager.
\item {\bf GET /types/:id} Return a list of all the attributes for the given type id.
\item {\bf POST /instances} Create a new instance from a given type and its attributes values. Redirects to the newly created instance.
\item {\bf GET /instances/:id} Return the type and the attribute values for the given instance.
\item {\bf PUT /instances/:id} Update one or more attributes for a given instance id (only attributes passed as in this call will be affected).											
\item {\bf DELETE /instances/:id } Delete the instance structure with the given id.						
\item {\bf POST /profiles} Create a new profile from an existing instance, a transformation and a policy. Redirects to the new profile.
\item {\bf GET /profiles/:id} Read profile by id: returns the type, the transformation, the policy and the attribute values for the given profile. 
\item {\bf DELETE /profiles/:id} Delete profile by id: delete the profile structure with the given id.	
\item {\bf POST /transformations}  Create a new transformation specifying the source type, the destination type and the attribute transformations to be performed.
\item {\bf GET /transformations/:id} Read transformation by id: return the source type, the destination type, and the attribute transformations for the given transformation.									
\item {\bf DELETE /transformations/:id} Delete transformation by id: delete the transformation structure with the given id.	
\item {\bf POST /policies} Create a new policy by specifying its use concept and validity limit.		
\item {\bf GET /policies/:id} Read policy by id: return the use concept and validity limit for the given policy.
\item {\bf DELETE /policies/:id} Delete policy by id: delete the policy structure with the given id.		
\item {\bf POST /peers}	Create peer: create a new peer by specifying its main entity.			
\item {\bf GET /peers/:id} Read peer: read the main entity and defined profiles for a peer.
\item {\bf PUT /peers/:id} Update peer profiles: update the defined profiles for a peer (replaces previous definitons).
\item {\bf DELETE /peers/:id} Delete peer: delete the peer structure with the given id.
\item {\bf POST /users}	Create user: create a new user by specifying the username (unique, passed as a string), password and main entity. Redirects to the new user. 											
\item {\bf GET /users/:id} Read user by id: read information stored in the user structure with the matching id.											
%\item {\bf GET /users/:username} Read user by username: read information stored in the user structure with the matching username.		
%\todo{clarify difference between the two calls above, seems dangerous as the endpoint is the same!}								
\item {\bf PUT /users/:id} Update user attributes, in particular it replaces the existing user password with a new one, specified in the request. %\todo{updates only pwd or user description?}
\item {\bf DELETE /users/:id} Delete user: deletes the user with the given id.
\end{itemize}
\subsubsection{Repository}
Not applicable, the PM is owned by the University of Trento and not disclosed. 