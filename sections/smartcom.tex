\subsection{Communication Middleware}
\subsubsection{Functionality and Features}
SmartCom provides communication functionality with human-based services and software-based services) on the other side. SmartCom provides low-level communication and control primitives that effectively virtualize the peers to HDA-CAS platform. It offers the asynchronous (message-based) communication functionality for interacting with dynamically-evolving collectives of peers both through native HDA-CAS applications, as well as through various third-party tools, such as Dropbox, Android devices, Twitter or email clients. More details can be found in~\cite{D7.2}.
\subsubsection{Implementation}
The CM is implemented in Java.
\subsubsection{Interfaces, Endpoints and Resources Exposed}
\todo{@Ogi: please check}
Message information is available through REST APIs:
\begin{itemize}
\item {\bf GET	/?type=:type\&subtype=<:subtype} Get the message information for a message with a specific type and subtype.
\item {\bf POST	/}	Add new message information for a message with a specific type and subtype. If there is already such a message information, it will be overridden with this information.
\end{itemize}
The Adapter REST Service exposes some Input adapters of SmartCom using REST:
\begin{itemize}
\item {\bf DELETE /adapter/:id} Deletes an adapter with the given id.
\item {\bf POST /adapter/rest} Create a new rest input adapter that listens on a specific port and uri for incoming messages.
\item {\bf POST /adapter/email} Create a new email input adapter. The request includes subject, host, username/password, type, subtype fields.
\item {\bf POST adapter/dropbox} Create a new dropbox input adapter. The request includes dropboxKey, dropboxFolder, fileName, type, subtype and conversationId fields.
\end{itemize}
The other functionality is available through Java functions:
\begin{itemize}
\item {\bf public Identifier send(Message message)} Send a message to a collective or a single peer. The method assigns an ID to the message and handles the sending asynchronously, i.e., it returns immediately and does not wait for the sending to succeed or fail. The receiver of the message is defined within the message, it can be a peer or a collective.
\item {\bf public Identifier addRouting(RoutingRule rule)} Add a special route to the routing rules (e.g., route input from peer A always to peer B). Returns the ID of the routing rule (can be used to delete it). The middleware will check if the rule is valID and throw an exception otherwise.
\item {\bf public RoutingRule removeRouting(Identifier routeId)} Remove a previously defined routing rule identified by an Id. As soon as the method returnsthe routing rule will not be applied any more. 
\item {\bf public Identifier addPushAdapter(InputPushAdapter adapter)} Creates a input adapter that will wait for push notifications or will pull for updates in a certain time interval. Returns the ID of the adapter.
 \item {\bf public Identifier addPullAdapter(InputPullAdapter adapter, long interval)} Creates a input adapter that will pull for updates in a certain time interval. Returns the ID of the adapter. 
 \item {\bf public Identifier addPullAdapter(InputPullAdapter adapter, long interval, boolean deleteIfSuccessful)} Creates a input adapter that will pull for updates in a certain time interval. Returns the ID of the adapter. The pull requests will be issued in the specified interval. If deleteIfSuccessful is set to true, the adapter will be removed in case of a successful execution, it will continue in case of a unsuccessful execution.
\item {\bf public InputAdapter removeInputAdapter(Identifier adapterId)}  Removes a input adapter from the execution. As soon as this method returns, the  adapter with the given ID will not be executed any more. 
\item {\bf public Identifier registerOutputAdapter(Class<? extends OutputAdapter> adapter)} Registers a new type of output adapter that can be used by the middleware to get in contact with a peer. The output adapters will be instantiated by the middleware on demand. 
\item {\bf public void removeOutputAdapter(Identifier adapterId)} Removes a type of output adapters. Adapters that are currently in use will be removed as soon as possible (i.e., current communication won’t be aborted and waiting messages in the adapter queue will be transmitted).
\item {\bf public Identifier registerNotificationCallback(NotificationCallback callback)} Register a notification callback that will be called if there are new input messages available.
\item {\bf public boolean unregisterNotificationCallback(Identifier callback)} Unregister a previously registered notification callback.
\end{itemize}  
\subsubsection{Repository}
The CM code can be found at: \url{https://gitlab.com/smartsociety/SmartCom/}
