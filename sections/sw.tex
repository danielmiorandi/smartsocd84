%!TEX root = ../main.tex
In this section we describe the software artifacts being produced, together with a description of their APIs.

\subsection{Orchestration Manager}
\subsubsection{Functionality and Features}
The OM provides collective composition and negotiation functionality. They can be invoked subsequently or jointly (in the latter case we talk about continuous orchestration, see~\cite{D6.2} for a detailed description. The OM takes as input task requests and outputs an agreed execution plan. 
\subsubsection{Implementation}
The OM implementation shipped with the toolbox is written in Javascript and based upon the node.js framework. It uses express for REST APIs, jade as node template and includes a mongoDB instance for persistency. Two versions are provided, supporting AskSmartSociety! and SmartShare applications. They can be easily reused to develop application-specific OMs.
\subsubsection{Interfaces, Endpoints and Resources Exposed}
\todo{taken from D2.3 - appendix}
For convenience we split the API into different sections.
B.1.1 Task Requests
We start with task requests. The most basic operations are listed in the following table:
 
verb
URI
POST
/applications/:app/taskRequests
GET
/applications/:app/taskRequests/?user=:user
GET
/applications/:app/taskRequests/:taskRequestID
HEAD
/applications/:app/taskRequests/:taskRequestID
GET
/applications/:app/taskRequests/:taskRequestID/v/:version
DELETE
/applications/:app/taskRequests/:taskRequestID
Create Task Request: POST /applications/:app/taskRequests This is the main URI where new task requests are posted. The JSON object describing the task request is expected in the body of the request. On success a platform call to the composition manager will be made.
Access Control. Any peer or user.
Success. Returns error code 201 together with
• a JSON document of the form {
28 of 57
http://www.smart-society-project.eu
Deliverable D2.3 ⃝c SmartSociety Consortium 2013-2017 data: aURI
}
where aURI is the URI where the client can retrieve (assuming authentication and access control policies have no issues) the latest version of the task request that has been posted, and optionally
• an ETag for the JSON object of the response.


Get Task Requests of User:
GET /applications/:app/taskRequests/?user=:user apart from authentication purposes (in the header).
Access Control. The peer user or an admin. Success. Returns error code 200 together with
• a JSON document of the form {
No parameters are expected
      data: [[userTaskRequestsURIs], [associatedETags]]
    }

    Get a Task Request:
GET /applications/:app/taskRequests/:taskRequestID No parameters are ex- pected apart from authentication information (if needed).
Access Control. The owner of the task request or an admin.
Success. Returns error code 200 together with the JSON document of the latest
version of the task request accompanied by the ETag of the document.
Failure. Returns an error code (403, 404) together with an optional error message.
Get the Head of a Task Request:
HEAD /applications/:app/taskRequests/:taskRequestID Similar to
GET /applications/:app/taskRequests/:taskRequestIDexceptthatthebodyreturned is empty. It just returns the ETag of the latest version of the task request to indicate if there has been a change to the document and thus we need to retrieve its latest version. The access control policy is similar as above; the owner of the task request or an admin can perform the operation.
Get a Specific Version of a Task Request:
GET /applications/:app/taskRequests/:taskRequestID/v/:version No param- eters are expected apart from authentication purposes (if needed).
Access Control. The owner of the task request or an admin.
Success. Returns error code 200 together with the specific version of the task request. Failure. Returns an error code together with an optional error message.
Delete a Task Request: DELETE /applications/:app/taskRequests/:taskRequestID
This is the main URI for deleting task requests. No parameters are expected apart from authentication information (in the header). A platform job is prepared and is posted to the deletion manager.
Access Control. The owner of the task request or an admin.

B.1.2 Tasks
Tasks are generated through composition (more on that below). The most basic operations related to them are listed in the following table:
 
verb
URI
GET
/applications/:app/tasks/:taskID
HEAD
/applications/:app/tasks/:taskID
GET
/applications/:app/tasks/:taskID/v/:version
PUT
/applications/:app/tasks/:taskID
Get a Specific Task: GET /applications/:app/tasks/:taskID Similar to GET /applications/:app/taskRequests/:taskRequestID but referring to tasks. No param- eters are expected apart from authentication information (if needed).
Access Control. The participants of the task or an admin.
Success. Returns error code 200 together with the JSON document of the latest
version of the task accompanied by the ETag of the document.
Failure. Returns an error code (403, 404) together with an optional error message.
Get the Head of a Task: HEAD /applications/:app/tasks/:taskID Similar as above but the body of the response is empty. Essentially this is an easy way for the clients to figure out if the resource has changed. Same access control policy as above.
Get a Specific Version of a Task:
GET /applications/:app/tasks/:taskID/v/:version No parameters are expected apart from authentication information (if needed).
Access Control. The participants of the task or an admin.
⃝c SmartSociety Consortium 2013-2017 31 of 57
⃝c SmartSociety Consortium 2013-2017 Deliverable D2.3 Success. Returns error code 200 together with the specific version of the task.
Failure. Returns an error code together with an optional error message.
Negotiate on a Task: PUT /applications/:app/tasks/:taskID The main call for negotiation which will trigger an additional platform call to the negotiation manager. Expects the new version of the document of the task taskID. A platform job for negotiation is prepared and is posted to the negotiation manager.
Access Control. The participants of the task or an admin.
Success. Returns error code 200 together with the new version of the task as is
dictated by the negotiation manager.
Failure. Returns an error code together with an optional error message.
B.1.3 Task Records
Task records are generated by the orchestrator once execution can start on a specific task. The most basic operations are listed in the following table:
 
verb
URI
GET
/applications/:app/taskRecords/:taskRecordID
HEAD
/applications/:app/taskRecords/:taskRecordID
GET
/applications/:app/taskRecords/:taskRecordID/v/:version
PUT
/applications/:app/taskRecords/:taskRecordID

Get a Specific Task Record:
GET /applications/:app/taskRecords/:taskRecordID No parameters are expected apart from authentication information (if needed).
Access Control. The participants of the task or an admin.
Success. Returns error code 200 together with the json document of the latest version
of the task record accompanied by the ETag of the document.

Get the Head of a Task Record:
HEAD /applications/:app/taskRecords/:taskRecordID No parameters are ex- pected apart from authentication information (if needed).
The body of the response is empty. This is another convenience function which allows an easy way for the clients to figure out if the resource has changed. Same access control policy and error codes as above.
Get a Specific Version of a Task Record:
GET /applications/:app/taskRecords/:taskRecordID/v/:version No parameters are expected apart from authentication purposes (if needed).
Access Control. The participants of the task or an admin.
Success. Returns error code 200 together with the specific version of the task. Failure. Returns an error code together with an optional error message.
Provide Execution Feedback:
PUT /applications/:app/taskRecords/:taskRecordID The main call for execu- tion which will trigger an additional platform call to the execution manager. Expects the new version of the task record document taskRecordID. A platform job for execution is prepared and is posted to the execution manager.
Access Control. The participants of the task or an admin.
Success. Returns error code 200 together with the new version of the task as dictated
by the execution manager.
Failure. Returns an error code together with an optional error message.

B.2 Composition Manager
The composition manager provides the following functionality:
verb
URI
POST
/applications/:app/compositions
GET
/applications/:app/compositions/:compositionID

Perform Composition: POST /applications/:app/compositions Expects the plat- form job with the description, for which the main ingredient is the new task request that has arrived on the platform.
Access Control. The orchestrator for the application app can make such a call.
Returns. The call always succeeds and generates a resource describing the outcome of composition. Upon completion it returns an error code 201 and the link to the document with the results of composition. Part of the description of the document with the results of the composition is the error code and message that is returned through the call POST /applications/:app/taskRequests to the client.
Get Composition Results:
GET /applications/:app/compositions/:compositionID No parameters are ex- pected.
Access Control. The orchestrator for the application app or an admin can make such a call.
Success. Returns error code 200, the JSON document with the description of the results of the composition together with the associated ETag for the document.
Failure. Returns an error code (e.g. 404 not found) together with an optional error message.
Comment. Normally such a call is expected to happen only once from the applica- tion orchestrator once the latter has received the 201 error code that the composition that was requested has been performed.

B.3 Negotiation Manager
The negotiation manager provides the following functionality.
Perform Negotiation: POST /applications/:app/negotiations Expects the plat- form job with the description, for which the main ingredient is the task on which negoti- ation is being performed.
Access Control. The orchestrator for the application app can make such a call.
Returns. The call always succeeds and generates a resource describing the outcome of negotiation. Upon completion it returns an error code 201 and the link to the document with the results of the negotiation. Part of the description of the document with the results of the negotiation is the error code and message that is returned through the call PUT /applications/:app/tasks/:taskID to the client.
Get Negotiation Results:
GET /applications/:app/negotiations/:negotiationID No parameters are expected.
Access Control. The orchestrator for the application app or an admin can make such a call.
Success. Returns error code 200, the JSON document with the description of the results of the negotiation together with the associated ETag for the document.
Failure. Returns an error code (e.g. 404 not found) together with an optional error message.
Comment. Normally such a call is expected to happen only once from the applica- tion orchestrator once the latter has received the 201 error code that the negotiation that was requested has been performed.
⃝c SmartSociety Consortium 2013-2017 35 of 57
 
verb
URI
POST
/applications/:app/negotiations
GET
/applications/:app/negotiations/:negotiationID



\subsubsection{Repository}
The OM code is available (Apache v.2) at: \url{https://gitlab.com/smartsociety/orchestration}

\subsection{Peer Manager}
\subsubsection{Functionality and Features}
The Peer Manager (PM) provides a peer-centered data store that maintains and manages information about human- or machine-based peers within a privacy-preserving framework.
The PM was designed with the objective of keeping the information owned by peers private. In order to provide a flexible management of information, the PM builds upon the notion of an entity-centric semantic enhanced model that defines an extensible set of entity schemas providing the templates for an attribute-based representation of peers’ characteristics~\cite{D4.2}. Additionally, the PM defines a storage and privacy protection model by adding privacy regulations and considerations. The PM is designed to comply with the recent General Data Protection Regulation (GDPR) (Regulation (EU) 2016/679).
\subsubsection{Implementation}
Not applicable, the PM is owned by University of Trento and implementation details are not made public.
\subsubsection{Interfaces, Endpoints and Resources Exposed}
% \todo{add from https://docs.google.com/spreadsheets/d/1UlRKshoALlSzmJYZVLkuzAGTUukdFKaONlKqPmd3RWo/edit#gid=0}
% \todo{@Ronald: please check the APIs \& also whether we can share them (deliverable is PU)}
We divide APIs in chapters. Let us start with the security ones:
\begin{itemize}
\item {\bf POST /security/login} Used for authentication. The call includes as parameters an identifier and a password, and returns, among the other data, a token to be used for further interactions with the platform.
\item {\bf POST /security/logout} Log the peer out of the PM.
\item {\bf GET /person} Read the person of a given logged user. 
\item {\bf GET /person\_sl?username=:username\&password=:password} Register a new human peer. Returns a peer ID. 
\end{itemize}
The second set of APIs cover management operations:
\begin{itemize}
\item {\bf GET /search}. General search machinery. Check~\cite{D4.3} for more explanations on the query language to be used. 	
\item {\bf POST /collectives} Create collective: create a collective structure from a set of usernames.
\item {\bf GET /collectives/identifier}	Return the usernames of all peers in a collective.				
\item {\bf DELETE /collectives/} Delete the collective with a given id.	
\item {\bf GET /types} Return the names and id of all the types defined in the peer manager.
\item {\bf GET /types/:id} Return a list of all the attributes for the given type id.
\item {\bf POST /instances} Create a new instance from a given type and its attributes values. 
\item {\bf GET /instances/:id} Return the type and the attribute values for the given instance.
\item {\bf PUT /instances/:id} Update one or more attributes for a given instance id (only attributes passed as in this call will be affected).											
\item {\bf DELETE /instances/:id } Delete the instance structure with the given id.						
\item {\bf POST /profiles} Create a new profile from an existing instance, a transformation and a policy.
\item {\bf GET /profiles/:id} Read profile by id: returns the type, the transformation, the policy and the attribute values for the given profile. 
\item {\bf DELETE /profiles/:id} Delete profile by id: delete the profile structure with the given id.	
\item {\bf POST /transformations}  Create a new transformation specifying the source type, the destination type and the attribute transformations to be performed.
\item {\bf GET /transformations/:id} Read transformation by id: return the source type, the destination type, and the attribute transformations for the given transformation.									
\item {\bf DELETE /transformations/:id} Delete transformation by id: delete the transformation structure with the given id.	
\item {\bf POST /policies} Create a new policy by specifying its use concept and validity limit.		
\item {\bf GET /policies/:id} Read policy by id: return the use concept and validity limit for the given policy.
\item {\bf DELETE /policies/:id} Delete policy by id: delete the policy structure with the given id.		
\item {\bf POST /peers}	Create peer: create a new peer by specifying its main entity.			
\item {\bf GET /peers/:id} Read peer: read the main entity and defined profiles for a peer.
\item {\bf PUT /peers/:id} Update peer profiles: update the defined profiles for a peer (replaces previous definitons).
\item {\bf DELETE /peers/:id} Delete peer: delete the peer structure with the given id.
\item {\bf POST /users}	Create user: create a new user by specifying the username(unique), password and main entity.											
\item {\bf GET /users/:id} Read user by id: read information stored in the user structure with the matching id.											
\item {\bf GET /users/:username}	Read user by username: read information stored in the user structure with the matching username.										
\item {\bf PUT /users/:id} Update user password: replaces the existing user password for the user with the given id.	
\item {\bf DELETE /users/:id} Delete user: deletes the user with the given id.
\end{itemize}
\subsubsection{Repository}
Not applicable, the PM is owned by the University of Trento and not disclosed. 

\subsection{Context Manager}
\subsubsection{Functionality and Features}
Context Manager (CTX): this component is the software/hardware embodiment of the 3-layer approach. In fact it consists of three different subcomponents:
1. Sensing devices
2. Modules for aggregating and merging attribute values from sensor data 3. High level models stored in each peer and connected via dedicated APIs
\subsubsection{Implementation}
Not applicable, the PM is owned by University of Trento and implementation details are not made public.

\subsubsection{Interfaces, Endpoints and Resources Exposed}

\subsubsection{Repository}
Not applicable, the PM is owned by the University of Trento. 

\subsection{Incentive Server}
\subsubsection{Functionality and Features}
% \todo{@Kobi/Avi: please check content and modify if appropriate}
The Incentive Server analyzes in real-time user interaction and behaviour data within a single application and recommends incentives for these  users. The IS includes learning methods, so that it can adapt its recommendations over time. The IS includes mechanisms for defining  incentive types and messages, for receiving and storing behavioral data, for applying different  algorithms on this behavioral data, for deciding on conditions that requires interventions and for  recommending interventions for the applications that use it. See~\cite{D5.4} for more details.
\subsubsection{Implementation}
The incentive server is written in Python, and uses the Django framework. 
\subsubsection{Interfaces, Endpoints and Resources Exposed}
% \todo{taken from https://docs.google.com/document/d/1lqdWNfu0RzNBStf5ddGREBBI1ftwA5azfV6bpvOImmY/edit}
\begin{itemize}
\item {\bf POST /login} Login Action. Used by external applications to obtain access to the intervention server (IS). Must be performed before any other operation access to the IS.
\item {\bf GET /api/incentive} Get all incentives types from the IS.
\item {\bf POST /api/incentive} Add a new incentive to the IS. Must specify scheme name/ID, type name/ID and conditions under which the incentive should be applied.
\item {\bf GET /getIncUser} Get the best incentive for a given user. This represents a pull operation by external application to obtain incentives for user.
\item {\bf GET /ask\_by\_date}  Get intervention list sent to users after the given date. 
\item {\bf GET /ask\_by\_id} Intervention or intervention list for the user with a given id.
\item {\bf GET /asl\_gt\_id} Intervention or intervention list for users with id greater than the given one.
\item {\bf GET /disratio} Returns the ratio between the number of users for which interventions were sent and the number of users for which no intervention was required.
\item {\bf POST /reminder} Send a reminder to a given collective after a specified timeout.
\item {\bf POST /invalidate/:collective\_id} Invalidate peers from a collective that is about to be reminded, so to make sure the reminder is sent only to those who have not replied yet.
\end{itemize}
The IS can be configured to work in push mode, sending incentives to an application; more details on how to do it are in~\cite{D5.4}. In the same document the functioning of the endpoint for retrieving behavioural data and inputting it to the IS is also described.
\subsubsection{Repository}
The incentive server implementation can be found at: \url{https://gitlab.com/smartsociety/IncentiveServer}

\subsection{Task Execution Manager}
\subsubsection{Functionality and Features}
\todo{@Agnes: please check}
The TEM is tasked with coordinating the execution of tasks, especially those involving `offline' actions by peers and collectives. The TEM acts as a monitoring platform and interacts with the Orchestration Manager and Context Manager/Peer Manager. It takes tasks agreed from the OM and instantiates the required monitors with the CM.
For each task to be monitored the Execution Monitor requires as an input from the OM:
\begin{itemize} 
\item A description of the task whose execution has to be monitored; 
\item The IDs of the peers involved in the execution
\end{itemize}
At the same time the TEM will also have access to:
\begin{itemize}
\item Sensory data related to the peers involved in the execution (from the CM) and;
\item Mid and high-level data derived by sensor fusion in time and user input
\end{itemize}
Using these two sources the TEM produces an output to inform the OM of:
\begin{itemize}
\item The progress of a given tasks; and
\item Possible deviations that occur during the execution of this task; major deviations should be properly expressed to allow the OM to react in a timely manner.
\end{itemize}
\subsubsection{Implementation}
The TEM is implemented in javascript using the node.js framework. Express is using for the REST API implementation and a MongoDB instance for storing data locally.

\subsubsection{Interfaces, Endpoints and Resources Exposed}
\begin{itemize}
\item {\bf GET /monitorTask} Get all the tasks which are being monitored by the TEM.
\item {\bf POST /monitorTask} Starts monitoring the execution of an agreed task by instantiating the relevant monitors. 
\item {\bf PUT /updateMonitor/:taskID} update the monitor of a task in TEM.
\item {\bf DELETE /terminateTaskMonitor/:taskID} Terminate the monitoring of a task.
\item {\bf GET /terminatedTasks} Get the list of all the terminated tasks.
\end{itemize}
\subsubsection{Repository}
\url{https://gitlab.com/smartsociety/taskexecutionmanager}

\subsection{Provenance Service}
\subsubsection{Functionality and Features}
Provenance is information about entities, activities, and people involved in producing a piece of data or thing, which can be used to form assessments about its quality, reliability or trustworthiness. The Provenance Service includes a specialised store for managing provenance records~\footnote{ProvStore, \url{https://provenance.ecs.soton.ac.uk/store/}} and an aggregator of provenance types able to generate provenance summary reports. More details on the vocabulary to be used for HDA-CAS can be found in~\cite{D2.4}. The provenance service allows to create and retrieve provenance templates and bindings, which are subsequently used for logging actions on the platform. 

\subsubsection{Implementation}
The provenance service is developed in Python and uses the Django framework.

\subsubsection{Interfaces, Endpoints and Resources Exposed}
% \todo{@Heather: please check}
\begin{itemize}
\item {\bf POST template/} Post a template: Posts a provenance template to the Provenance Store (https://provenance.ecs.soton.ac.uk/store/). The data posted must contain key value pairs for prov and template\_name.
\item {\bf GET template/:template/} Get a template: Returns a JSON object detailing the template's id, name, version and a URI linking to where the template is stored in the Provenance Store.
\item {\bf POST binding/} Post a provenance binding to the Provenance Store. The data posted must contain key value pairs for prov, template\_name, and binding\_name, it can optionally contain a version\_id.  When a binding is submitting without specifying a version, it is associated with the latest version of a template with template\_name.
\item {\bf GET binding/:binding/} Get a binding: returns a JSON object detailing the binding's id, name, and a URI linking to where the binding is stored in the Provenance Store.
\item {\bf POST template/name/} Get all versions of a template: returns a JSON object containing all template versions with a specified name.
\end{itemize}

\subsubsection{Repository}
The {\tt prov} package (a library for W3C provenance data model) is used by the PS and can be downloaded (MIT License) from \url{https://pypi.python.org/pypi/prov}.
The PS codebase can be found at: \url{https://gitlab.com/smartsociety/submitbindings}


\subsection{Reputation Service}
\subsubsection{Functionality and Features}
The reputation service provides storage for and access to feedback and reputation reports, for SmartSociety applications. See~\cite{D2.4} for additional details.
\subsubsection{Implementation}
The RS is implemented in Python and uses the Django framework and the prov library.
\subsubsection{Interfaces, Endpoints and Resources Exposed}
% \todo{@Heather: please check}
The API currently describes three resources, applications, feedback, and reputation, which are documented below. 
\begin{itemize}
\item {\bf GET application/:app/subject/byURI/:subject\_uri/} Retrieve a subject’s information by its URI. The subject uri must be percentage encoded. The response object includes a list of feedback reports about the subject and the current reputation report for a subject.
\item {\bf GET application/:app/subject/:subject/} Retrieve information about the subject with the given ID. The response object includes the subject’s URI, a list of feedback reports about the subject and the the current reputation report for a subject.
\item {\bf POST application/:app/feedback/} Save a new feedback report. 
\item {\bf GET application/:app/reputation/:reputation/} Retrieve the raw reputation report with the given ID.
\item {\bf application/:app/opinionOf/:author/aboutSubject/:subject/} Retrieve the raw reputation report about a subject authored by an author.
\end{itemize}

\subsubsection{Repository}
The reputation service implementation can be found at: \url{https://gitlab.com/smartsociety/reputationservice2.0/}

\subsection{Communication Middleware}
\subsubsection{Functionality and Features}
SmartCom provides communication functionality between a Hybrid Diversity-Aware Collective Adaptive System platform (HDA-CAS) on one side, and ICUs (Individual Computing Units, i.e., human-based services and software-based services) on the other side. SmartCom provides low-level communication and control primitives that effectively virtualize the peers to HDA-CAS platform. It offers the asynchronous (message-based) communication functionality for interacting with dynamically-evolving collectives of peers both through native HDA-CAS applications, as well as through various third-party tools, such as Dropbox, Android devices, Twitter or email clients.
\subsubsection{Implementation}
The CM is implemented in java.
\subsubsection{Interfaces, Endpoints and Resources Exposed}
REST API

GET	`/?type=<type>&subtype=<subtype>`	Get the message information for a message with a specific type and subtype
POST	`/`	Add new message information for a message with a specific type and subtype
GET /?type=<type>&subtype=<subtype>

Returns the message information for a message with a specific type and subtype.

Parameter:
type required - defines the type of the message subtype required - defines the subtype of the message

Respond:
200 message information instance is returned in the body
404 there is no message information for this type and subtype combination

POST /

Create a new message information for a given type and subtype (encoded in the body). If there is already such a message information, it will be overridden with this information.

Parameter:
body required - instance of a message information JSON object

Respond:
200 resource has been created and the created message information is returned in the body

/**
     * Send a message to a collective or a single peer. The method assigns an ID to the message and
     * handles the sending asynchronously, i.e., it returns immediately and does not wait for the
     * sending to succeed or fail. Errors and exceptions thereafter will be sent to the Notification
     * Callback. Optionally, receipt acknowledgements are communicated back through the Notification
     * Callback API.
     *
     * The receiver of the message is defined within the message, it can be a peer or a collective.
     *
     * @param message Specifies the message that should be handled by the middleware.  The receiver of the message is
     *                defined by the message.
     * @return Returns the internal ID of the middleware to track the message within the system.
     * @throws CommunicationException a generic exception that will be thrown if something went wrong
     *                                in the initial handling of the message.
     */
    public Identifier send(Message message) throws CommunicationException;

    /**
     * Add a special route to the routing rules (e.g., route input from peer A
     * always to peer B). Returns the ID of the routing rule (can be used to delete it).
     * The middleware will check if the rule is valID and throw an exception otherwise.
     *
     * @param rule Specifies the routing rule that should be added to the routing rules of the middleware.
     * @return Returns the middleware internal ID of the rule
     * @throws InvalidRuleException if the routing rule is not valid.
     */
    public Identifier addRouting(RoutingRule rule) throws InvalidRuleException;

    /**
     * Remove a previously defined routing rule identified by an Id. As soon as the method returns
     * the routing rule will not be applied any more. If there is no such rule with the given Id,
     * null will be returned.
     *
     * @param routeId The ID of the routing rule that should be removed.
     * @return The removed routing rule or null if there is no such rule in the system.
     */
    public RoutingRule removeRouting(Identifier routeId);

    /**
     * Creates a input adapter that will wait for push notifications or will pull for updates in a
     * certain time interval. Returns the ID of the adapter.
     *
     * @param adapter Specifies the input push adapter.
     * @return Returns the middleware internal ID of the adapter.
     */
    public Identifier addPushAdapter(InputPushAdapter adapter);

    /**
     * Creates a input adapter that will pull for updates in a certain time interval.
     * Returns the ID of the adapter. The pull requests will be issued in the specified
     * interval until the adapter is explicitly removed from the system.
     *
     * @param adapter  Specifies the input pull adapter
     * @param interval Interval in milliseconds that specifies when to issue pull requests. Can’t be zero or negative.
     * @return Returns the middleware internal ID of the adapter.
     */
    public Identifier addPullAdapter(InputPullAdapter adapter, long interval);

    /**
     * Creates a input adapter that will pull for updates in a certain time interval.
     * Returns the ID of the adapter. The pull requests will be issued in the specified
     * interval. If deleteIfSuccessful is set to true, the adapter will be removed in case of
     * a successful execution, it will continue in case of a unsuccessful execution.
     *
     * @param adapter  Specifies the input pull adapter
     * @param interval Interval in milliseconds that specifies when to issue pull requests. Can’t be zero or negative.
     * @param deleteIfSuccessful delete this adapter after a successful execution
     * @return Returns the middleware internal ID of the adapter.
     */
    public Identifier addPullAdapter(InputPullAdapter adapter, long interval, boolean deleteIfSuccessful);

    /**
     * Removes a input adapter from the execution. As soon as this method returns, the
     * adapter with the given ID will not be executed any more. It returns the requested
     * input adapter or null if there is no adapter with such an ID in the system.
     *
     * @param adapterId The ID of the adapter that should be removed.
     * @return Returns the input adapter that has been removed or nothing if there is no such adapter.
     */
    public InputAdapter removeInputAdapter(Identifier adapterId);

    /**
     * Registers a new type of output adapter that can be used by the middleware to get in contact with a peer.
     * The output adapters will be instantiated by the middleware on demand. Note that these adapters are required
     * to have an @Adapter annotation otherwise an exception will be thrown.
     * In case of a stateless adapter, it is possible that the adapter will be instantiated immediately. If
     * any error occurs during the instantiation, an exception will be thrown
     *
     * @param adapter The output adapter that can be used to contact peers.
     * @return Returns the middleware internal ID of the created adapter.
     * @throws CommunicationException if the adapter could not be handled, the specific reason is embedded in
     *                                the exception.
     * @see at.ac.tuwien.dsg.smartcom.adapter.annotations.Adapter
     */
    public Identifier registerOutputAdapter(Class<? extends OutputAdapter> adapter) throws CommunicationException;

    /**
     * Removes a type of output adapters. Adapters that are currently in use will be removed
     * as soon as possible (i.e., current communication won’t be aborted and waiting messages
     * in the adapter queue will be transmitted).
     *
     * @param adapterId Specifies the adapter that should be removed.
     */
    public void removeOutputAdapter(Identifier adapterId);

    /**
     * Register a notification callback that will be called if there are new input messages
     * available.
     *
     * @param callback callback for notification
     * @return returns the identifier of the callback (can be used to remove it)
     */
    public Identifier registerNotificationCallback(NotificationCallback callback);

    /**
     * Unregister a previously registered notification callback.
     *
     * @param callback callback for notification
     */
    public boolean unregisterNotificationCallback(Identifier callback);
\subsubsection{Repository}
\url{https://gitlab.com/smartsociety/SmartCom/}


\subsection{Monitoring}
\subsubsection{Functionality and Features}
The goal of the monitoring component is to enable system administrators to monitor the liveliness of the SmartSociety platform components, possibly distributed across multiple servers. Target users of the functionality exposed are therefore system administrators and developers of SmartSociety-based CAS enabled applications and services. The main expected usage is for troubleshooting in case of platform malfunctioning, alerts and warnings. In the long term, it can enable the deployment of self-healing mechanisms.


\subsubsection{Implementation}
The implementation of the monitoring component is based upon three basic components:
\begin{itemize}
\item Modules gathering and publishing the relevant information;
\item Modules consuming the monitoring related information;
\item Information dissemination infrastructure.
\end{itemize}
% In particular, local agents collect specific information regarding the liveliness and health status of each component and communicate with the infrastructure via lightweight clients. The monitoring infrastructure provides scalable support to the collection of local agents feeds. A monitoring dashboard represents the main consumer of monitoring data generated by the agents and dispatched through the monitoring infrastructure.
The implementation shipped with the SmartSociety toolkit~\cite{D8.3} is based on the logstash open source framework\footnote{{\tt http://logstash.net/}}, which presents very good support for collection of logs in various schemas/formats, and has a large number of plugins available for the most widely used commercial frameworks. The logstash forwarders accept data using {\tt gerf} protocol on UDP. The logs are then forwarded through secured channel to the main logstash instance.
 The usage of logstash is coupled with elasticsearch for indexing and persistence. Aggregated and curated logs are stored in the elastic database and can be queried via standard interfaces, enabling technical supporting partners to develop their own ad hoc monitoring dashboard or to integrate with legacy ones. The default option for SmartSociety is to use Kibana\footnote{{\tt https://www.elastic.co/products/kibana}}, a flexible dashboard which supports seamless integration with elasticsearch and presents basic, yet sufficient analytics functionality. The metrics and specific charts can be configured dynamically by the administrator of the platform.
\subsubsection{Interfaces, Endpoints and Resources Exposed}
The Monitoring component functionality is accessible through the Kibana GUI. Elasticsearch Search APIs can be used for integration with legacy visualization dashboards. 
\subsubsection{Repository}
Not applicable.
\todo{Tommaso: create project in gitlab and add docker-compose to the repo}












% At the moment the platform mainly consists of a runtime allowing a Smart Society application to manage the submission of tasks by the user application. The platform provides also some library (against which the application is compiled) to operate with the current components (SmartCom, Orchestration Manager, Peer Manager, Provenance Service). The code is in a private repository at \url{https://gitlab.com/smartsociety/appruntime}.

% The document also presented the overall execution context of the programming model and the corresponding high-level language constructs used to manipulate them. The lan- guage constructs represent an API for the implementation of the programming model as a Java library executable on the SmartSociety platform runtime being developed in WP8.


% The final version of the application runtime will be heavily dependent on the Programming Framework, whose model has been defined in~\cite{D7.2}. The current prototype has been developed with the following aims: 
% % has been created before this effort has three goals:
% \begin{itemize}
% 	\item To identify what the runtime should expect from a generic application;
%     \item To define a first high level API to be provided to application developers;
% 	\item To test the interaction of the platform with the different components.
% \end{itemize}

A Smart Society Application has its own identifier, generated during the registration phase. Currently, the application code has to be written by a developer directly in Java; in the final version of the platform, the code will be generated dynamically through the programming framework. %Each time a task is submitted to the application a new application instance is created, each instance with its own state.

\subsection{Runtime}
Each application is expected to run in its own process by the SmartSociety runtime. The application will be provided by runtime with a SmartCom and a Orchestrator instances. In order to interact with external entities (such as peers or user applications), the runtime exposes three kinds of resources:

\begin{itemize}
\item {\bf POST:/task/:applicationId/} To submit tasks, the runtime will ask the application to create an \textit{instance initializer} that will take care of the setup phase of the task. The application is expected to carry out all the operation that can be performed before interacting with the peers. The posted data is domain--dependent, and it will be serialized and managed by the application at the moment of instance creation. The runtime associated to the instance (hence the task) creates an identifier that is returned as response for further usage.

\item {\bf GET:/task/:applicationId/} This resource is used to query the status of one or multiple tasks. Query parameters can be used to specify a filter on the tasks of interest. 
%all or part of the tasks of the application, query parameters can be used in the query and they will be managed by the application that might filter which tasks to show based non them. Note that t
The status format for each task is just required to be a valid json node (either a text or more complex structures) and it is completely domain--dependent. %, in fact the SmartSociety application can send whatever information the user application requires.

\item {\bf GET:/task/:applicationId/:taskId} Used to retrieve the status of a specific task. As for the previous point both query parameters and the response are domain--dependent.

\item {\bf POST:/message/:applicationId/} Used by the peer applications to communicate back with the application. To use this endpoint the application must have previously contacted the peer with a message containing a given conversation identifier. Such identifier, which is passed as a specific field in the message body, is used  by the runtime to dispatch the message to the correct application instance. The content of the message is then handled by the task runner instance that changes its state according to the information received.
\end{itemize}

In v.2.0 of the platform, the application developer is expected to provide certain functionalities. This includes methods for:
\begin{itemize}
 	\item Creating a new task runner that will take care of setting up and carry out the task submitted to the application;
 	\item Retrieve the active task runners according to some query parameters, this will be used by the query endpoint.
\end{itemize}
The task runner is the instance of an application-specific class, which implements a simple interface allowing the runtime to start the execution and to ask for its status.


\subsection{Component Library}
The component library allows the application development to be abstracted from the actual implementation of the components.
Component wrappers provide a simple interface to the component integrated with the runtime, so that the developer is shielded from minor changes in components. %, and finally testing will be easier as well.

In this section we briefly describe the functionality % give some example of functionalities 
provided by the component library. % divided by component.
\subsubsection{SmartCom}
The developer does not need to interact directly with \mdl: 
a method is provided for sending a message to a given collective, the programmer is required to provide a specific handler for handling answers to the message. The runtime will take care of receiving answers through the specific REST endpoint 
%(using an ad-hoc) 
and, by using the conversation ID, it will route the answer to the correct handler. The platform provides also an adapter for sending out Android notifications or for contacting peers trough a REST endpoint. Other supported adapters (Dropbox, Email, ...) can be added easily. 
%and adding them is straightforward.

\subsubsection{Orchestration Manager}
When the application is launched %At the application start time 
an orchestration manager (OM) is also executed. Communication with the OM works trough a REST API (a generalized version of the one presented in~\cite{D6.2}). % but this part is hidden to the developer, 
The library supports the %makes easy for the developer to make the 
following requests:
\begin{itemize}
	\item Submit a task request for composition;
	\item Retrieve a specific task request or task;
	\item Wait for a negotiable task;
	\item Accept a task on behalf of a peer;
	\item Perform a trivial negotiation with explicit agreement; 
	\item Wait for an agreed task.
\end{itemize}

%%DM: removed 31/7, too negative
%Because of limitations of the current implementation of the PM the composition consists in retrieving all the peer registered to the application; this query is %, such a query is 
%performed directly by the OM.

\subsubsection{Peer Manager}
The library allows the easy creation of a collective given a collection of peers. The identifier of the collective is the only information needed by SmartCom to carry out communications with all the relevant peers. %transparently to the developer.

\subsubsection{Provenance}
The runtime provides easy methods for logging on the provenance store the generic (i..e, independent from the application domain) part of the provenance graph. This includes actions such as, e.g., creation of a task, creation of a collective etc. The domain-dependent part of the provenance graph shall be specified by the application developer.% indipendent from the application domain.
Some helper function is provided for binding specific data whose model cannot be known a priori.
The communication with provenance store (happening through a REST API) is hidden to the developer by the library.

\subsection{Monitoring Framework}
%\todo{In the repository there is nothing like that}
The monitoring framework is responsible for the monitoring of the overall SmartSociety platform and components. It acts as a central collection point for any information which is considered important to ensure the proper functioning of the platform. More details are reported in App.~\ref{app:monitoring}.

% More in detail, it allows to:
% \begin{itemize}
% \item dynamically collect any kind of monitoring information relevant to monitor the proper functioning pf the platform and its performance. Such information is permanently stored in a non-relational database and available at any time for inspecting specific platform behaviors.
% \item visualize such information in real-time by means of interactive dashboards, or query the collected data via dedicated APIs. In particular, through the Monitoring framework it is possible to create and configure specific visualizations starting from the data that has been collected. There can be multiple visualizations, each one geared towards a specific platform KPI or informative visualization.
% \end{itemize} 

% %first, it allows to permanently store the data that is collected from the various COMPOSE components. The information can then be explored over time in order to either analyse the performance of platform components, or identify the causes of a specific malfunctioning. 
% %Second, it allows to visualize both real-time, as well historical data. In particular, through the Monitoring dashboard it is possible to create and configure specific visualizations starting from the data that is collected. There can be multiple visualizations, each one geared towards a specific platform KPI or information.
% %The COMPOSE platform administrator is expected to be the key utilizer of the Monitoring Dashboard. The current implementation does not support different visualizations based on user role.
% The Monitoring Dashboard is based on the following architecture and components:


% \begin{figure}[!hbt]
% \centering
% \includegraphics[width=0.8\textwidth]{figs/monitoring.pdf}
% \caption{Monitoring framework architecture.}
% \end{figure}

% The monitoring framework is based on Logstash (http://logstash.net/), which is a tool for managing events and logs. Logstash can be used to collect logs, parse them, and store them for later use (like, for search and visualization). Speaking of searching, Logstash comes with a web interface for searching and drilling into all of the logs collected over time.
% It is possible to define what information shall be permanently store and processed by the monitoring framework. In particular, it is possible to integrate:
% \begin{itemize}
% \item System logs: these logs correspond to logs, which are generated by the various system components such as, e.g., web servers, application servers. 
% \item Application logs: specific logs that are produced by applications, and require a constant integration for debugging and monitoring purposes.
% \item Monitoring information: any agent that can be configured to deliver data to the Logstash infrastructure.
% \end{itemize}

% In all three cases, a Logstash shipper is used to connect the specific source of data to Logstash. Specific shippers already exists for some widely use system components such as, e.g., web servers, databases, etc., while custom shippers can be created for specific cases. In the case of SmartSociety, we created a dedicated shipper to collect the events produced by the various platform components.\\
% The following component is a Redis Broker. This is an optional component that can be used in order to scale the system to large volumes of events and data. Based on Redis, data is indexed in order to prepare it for optimal searching and querying. Once the data is indexed, it is stored in an ElasticSearch cluster for storage and search. 
% Starting from the data stored in ElasticSearch, it is possible to build queries on scale to explore the collected data. We used Kibana (https://www.elastic.co/products/kibana) as the tool to create and visualize queries on the collected data. Kibana is fully integrated with ElastichSearch, and allows easily explore and give sense to large volumes of data.
% In addition, ElastichSearch provides APIs for querying and extracting the data stored in the platform. This can be helpful in the case aggregated views such as, e.g., monthly reports, are needed.
% The following Figure provides an example of dashboard created over Kibana. The metrics and specific charts can be configured dynamically by the administrator of the platform.

% \begin{figure}[!hbt]
% \centering
% \includegraphics[width=0.8\textwidth]{figs/kibana.pdf}
% \caption{Example of the monitoring framework web interface.}
% \end{figure}


\subsection{Expected evolution}

In the future version of the platform each application and its runtime will run in separate containers. This will provide isolation among different applications and will support faster application deployment and portability across machines. A service for routing the requests from general endpoints to the right application runtime will be included.

The way the runtime will evolve and its interaction with applications are highly dependent on the outcome of the programming model and programming framework specification efforts currently ongoing within WP7.

In terms of integration of other components, the roadmap foresees to integrate in the upcoming months the context manager (which will interact strictly with the peer manager), the reputation maanager, the task execution manager and the incentive server.